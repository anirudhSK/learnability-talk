\begin{frame}
\frametitle{Learning network protocols despite imperfect assumptions}
\begin{centering}
Do we need to know the exact topology at design time?
\end{centering}
\end{frame}

\begin{frame}
\frametitle{When the model is wrong about the topology}
\large
\begin{centering}

\begin{itemize}
\item<2-> What if we train on a one-bottleneck topology\\
\begin{centering}
\includegraphics[width=0.8\textwidth]{onelink.pdf}
\end{centering}
\item<3-> But, test on a two-bottleneck topology\\
\begin{centering}
\includegraphics[width=0.8 \textwidth]{twolink.pdf}
\end{centering}
\end{itemize}
\end{centering}
\end{frame}

\begin{frame}
\frametitle{When the model is wrong about the topology}
\begin{centering}

\noindent \only<1>{\includegraphics[width=3.4 in]{multilink-background.pdf}}\only<2>{\includegraphics[width=3.4 in]{multilink-cubic.pdf}}\only<3>{\includegraphics[width=3.4 in]{multilink-omniscient.pdf}}\only<4>{\includegraphics[width=3.4 in]{multilink-codel.pdf}}\only<5>{\includegraphics[width=3.4 in]{multilink-twolink.pdf}}\only<6>{\includegraphics[width=3.4 in]{multilink-onelink.pdf}}
\only<7>{Simplifying a two-link network to a single-link network only modestly hurts performance}
\end{centering}
\end{frame}

