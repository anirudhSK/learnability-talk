\documentclass[svgnames]{beamer}
\usetheme{Dresden}
\usecolortheme{beaver}
\usepackage{textcomp}
\usepackage{color}
\usepackage{colortbl}
\usepackage{pbox}
\title{An experimental study of the learnability of congestion control}
\author{Anirudh~Sivaraman, Keith~Winstein, Pratiksha~Thaker, Hari~Balakrishnan}
\institute{MIT CSAIL\vspace{\baselineskip}}
%\\\textcolor{DarkBlue}{http://web.mit.edu/anirudh/www/sdn-data-plane.html}}
%\date{}


\begin{document}

\begin{frame}

\titlepage

\end{frame}

\begin{Large}
\begin{frame}
\frametitle{Congestion-control protocols today}
\begin{itemize}
\item<2-> Set of human-designed rules at end points (and switches)
\item<3-> Protocols implicitly assume certain network conditions
\end{itemize}
\end{frame}

\begin{frame}
\frametitle{Assumptions break with time (figure out how to tie these better to the questions we do answer)}
\begin{itemize}
\item Lost throughput due to stochastic loss 
\item Bufferbloat when queues are incorrectly sized
\item Incast in datacenters
\end{itemize}
\end{frame}

\begin{frame}
\frametitle{Our work}
\begin{itemize}
\item<1-> What assumptions matter and what don't?
\item<2-> Can we formalize this design process ...
\item<3-> to quantify: how easy is it to “learn” a network protocol to achieve desired goals, given
an imperfect set of assumptions: the network model?
\end{itemize}
\end{frame}

\begin{frame}
\frametitle{Outline}
\begin{itemize}
\item Formalize learnability in the context of congestion control
\item Use it to answer:
\begin{itemize}
\item Do we need to know the link speed exactly?
\item Do we need to know the degree of multiplexing exactly?
\item What is the cost of backwards compatibility?
%\item Do we need to know the topology exactly?
%\item What is the cost of application diversity?
\end{itemize}
\end{itemize}
\end{frame}

\begin{frame}
\frametitle{Approach}
\begin{itemize}
\item Specify \textit{training scenarios}
\begin{itemize}
\item Topology
\item Locations of senders and receivers
\item Application workload
\item Buffer size and queuing discipline 
\end{itemize}
\item Specify an \textit{objective function}
\item Synthesize protocol automatically
\item Evaluate on \textit{testing scenarios} inside ns-2
\end{itemize}
\end{frame}

\begin{frame}
\frametitle{Formalizing the design process}
\begin{itemize}
\item Find best protocol, given an imperfect network model
\item The problem is hard to solve in general
\item<2-> Compromise: Rely on Remy\footnote<2->{Keith Winstein and Hari Balakrishnan, \textbf{TCP ex Machina: Computer-Generated Congestion Control}, \textit{SIGCOMM 2013}} instead
\end{itemize}
\end{frame}

\begin{frame}
\frametitle{TODO: Make succinct: The protocol synthesis procedure}
\begin{itemize}
\item Represent protocol as a range-based lookup table from \textit{state} to \textit{action}
\item State: Designer-specified congestion signals
\item Action: increasing/decreasing window or a new transmission rate
\end{itemize}
\end{frame}

\begin{frame}
\frametitle{TODO The protocol synthesis procedure: snarf Keith's figures from SIGCOMM 2013}
\begin{itemize}
\item Start off with a constant function: same action regardless of state
\item Optimize this action to maximize objective
\item Bisect lookup table based on state usage retaining the same action
\end{itemize}
\end{frame}

\input caveats

\input optimality

\input linkspeed

\input multiplexing

\input compatibility



\begin{frame}
\frametitle{Related Work}
\begin{itemize}
\item Probably approximately correct learning \textbf{(Valiant, 1984)})
\item Transfer learning \textbf{(Pan \& Yang, 2010)}
\item Stochastic network utility maximization \textbf{(Yi \& Chiang, 2008)}
\end{itemize}
\end{frame}

\begin{frame}
\frametitle{Ongoing and future work}
\begin{itemize}
\item<1-> Model mismatch between optimizer/simulator and reality
\item<2-> Computer-generated congestion control for the datacenter
\end{itemize}
\end{frame}

\begin{frame}
\frametitle{Conclusion and outlook}
\begin{itemize}
\item<1-> Application needs and network constraints keep changing ...
\item<2-> ... so new transport protocols are continuosly developed.
\item<3-> Can we automate this process like other disciplines have done?
\end{itemize}
\end{frame}

\end{Large}

\begin{frame}
\frametitle{Backup slides}
\end{frame}
\input topology
\input diversity
\end{document}
