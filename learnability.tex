\documentclass[svgnames]{beamer}
\usetheme{Dresden}
\usecolortheme{beaver}
\usepackage{textcomp}
\usepackage{color}
\usepackage{colortbl}
\usepackage{pbox}
\setbeamerfont{page number in head/foot}{size=\large}
\setbeamertemplate{footline}[frame number]
\title{An experimental study of the learnability of congestion control}
\author{Anirudh~Sivaraman, Keith~Winstein, Pratiksha~Thaker, Hari~Balakrishnan}
\institute{MIT CSAIL\vspace{\baselineskip}}
%\\\textcolor{DarkBlue}{http://web.mit.edu/anirudh/www/sdn-data-plane.html}}
%\date{}


\begin{document}

\begin{frame}

\titlepage

\end{frame}

\begin{Large}
\begin{frame}
\frametitle{Computer-generated congestion control (Winstein et al 2013)}
\begin{itemize}
\item A model of the network
\item Objective function
\item Protocol
\item TODO: Make this a figure
\end{itemize}
\end{frame}

\input optimality

\begin{frame}
\begin{itemize}
\frametitle{What if the model is imperfect?}
\item TODO: Bring up Remy figure again but distinguish training and testing models.
\end{itemize}
\end{frame}

\begin{frame}
 how easy is it to “learn” a network protocol to achieve desired goals, given
an imperfect set of assumptions, i.e., the network model?
\end{frame}

\input linkspeed

\input multiplexing

\input topology 

\input compatibility

\begin{frame}
\frametitle{Related Work}
\begin{itemize}
\item Probably approximately correct learning \textbf{(Valiant, 1984)})
%\item Transfer learning \textbf{(Pan \& Yang, 2010)}
\item Stochastic network utility maximization \textbf{(Yi \& Chiang, 2008)}
\end{itemize}
\end{frame}

\begin{frame}
\frametitle{Ongoing and future work}
\begin{itemize}
\item<1-> What does automated protocol synthesis teach us about protocol design?
\item<2-> Model mismatch between optimizer/simulator and reality
\item<3-> Automated synthesis of congestion control for the datacenter
\end{itemize}
\end{frame}

\begin{frame}
\frametitle{Conclusion and outlook}
\begin{itemize}
\item<1-> Application needs and network conditions evolve ...
\item<2-> ... so we design new transport mechanisms.
\item<3-> How hard is it to automate this process like other disciplines have done?
\item<4-> Paper available at: http://web.mit.edu/anirudh/www/learnability-sigcomm2014.pdf
\end{itemize}
\end{frame}

\end{Large}

\begin{frame}
\frametitle{Backup slides}
\end{frame}
\input remy
\input diversity
\end{document}
