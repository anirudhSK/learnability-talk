\documentclass[svgnames]{beamer}
\usetheme{Dresden}
\usecolortheme{beaver}
\usepackage{textcomp}
\usepackage{color}
\usepackage{colortbl}
\usepackage{pbox}
\setbeamerfont{page number in head/foot}{size=\large}
\setbeamertemplate{footline}[frame number]
\title{An experimental study of the learnability of congestion control}
\author{Anirudh~Sivaraman, Keith~Winstein, Pratiksha~Thaker, Hari~Balakrishnan}
\institute{MIT CSAIL\vspace{\baselineskip}}
%\\\textcolor{DarkBlue}{http://web.mit.edu/anirudh/www/sdn-data-plane.html}}
%\date{}


\begin{document}

\begin{frame}

\titlepage

\end{frame}

\begin{Large}
\begin{frame}
\frametitle{Congestion-control protocols today}
\begin{itemize}
\item<2-> Set of rules at hosts (and switches)
\item<3-> Implicitly assume certain network conditions
\item<4-> Implicit performance goals
\end{itemize}
\end{frame}

\begin{frame}
\frametitle{Assumptions break with time}
\begin{itemize}
\item Lost throughput due to stochastic loss 
\item Bufferbloat when queues are incorrectly sized
\item Incast in datacenters
\end{itemize}
\end{frame}

\begin{frame}
\frametitle{Our work}
\begin{itemize}
\item<1-> What assumptions matter and what don't?
\item<2-> Can we make goals and assumptions explicit ...
\item<3-> ... to quantify: how easy is it to “learn” a network protocol to achieve desired goals, given
an imperfect set of assumptions, i.e., the network model?
\end{itemize}
\end{frame}

\begin{frame}
\frametitle{Outline}
\begin{itemize}
\item Formalize learnability in the context of congestion control
\item Use it to answer:
\begin{itemize}
\item Do we need to know the exact link speed?
\item Do we need to know the exact degree of multiplexing?
\item What is the cost of backwards compatibility?
%\item Do we need to know the topology exactly?
%\item What is the cost of application diversity?
\end{itemize}
\end{itemize}
\end{frame}

\begin{frame}
\frametitle{Approach}
\begin{itemize}
\item Specify \textit{training scenarios}
\begin{itemize}
\item Topology
\item Locations of senders and receivers
\item Application workload
\item Buffer size and queuing discipline 
\end{itemize}
\item Specify an \textit{objective function}
\item Synthesize protocol automatically
\item Evaluate on \textit{testing scenarios} inside ns-2
\end{itemize}
\end{frame}

\begin{frame}
\frametitle{Formalizing the design process}
\begin{itemize}
\item Find best protocol, given an imperfect network model
\item The problem is hard to solve in general
\item<2-> Compromise: Rely on Remy\footnote<2->{Keith Winstein and Hari Balakrishnan, \textbf{TCP ex Machina: Computer-Generated Congestion Control}, \textit{SIGCOMM 2013}} instead
\end{itemize}
\end{frame}

\input caveats

\input remy

\input optimality

\input linkspeed

\input multiplexing

\input compatibility

\begin{frame}
\frametitle{Related Work}
\begin{itemize}
\item Probably approximately correct learning \textbf{(Valiant, 1984)})
%\item Transfer learning \textbf{(Pan \& Yang, 2010)}
\item Stochastic network utility maximization \textbf{(Yi \& Chiang, 2008)}
\end{itemize}
\end{frame}

\begin{frame}
\frametitle{Ongoing and future work}
\begin{itemize}
\item<1-> What does automated protocol synthesis teach us about protocol design?
\item<2-> Model mismatch between optimizer/simulator and reality
\item<3-> Automated synthesis of congestion control for the datacenter
\end{itemize}
\end{frame}

\begin{frame}
\frametitle{Conclusion and outlook}
\begin{itemize}
\item<1-> Application needs and network conditions evolve ...
\item<2-> ... so we design new transport mechanisms.
\item<3-> How hard is it to automate this process like other disciplines have done?
\item<4-> Paper available at: http://web.mit.edu/anirudh/www/learnability-sigcomm2014.pdf
\end{itemize}
\end{frame}

\end{Large}

\begin{frame}
\frametitle{Backup slides}
\end{frame}
\input topology
\input diversity
\end{document}
