\documentclass[svgnames]{beamer}
\usetheme{Dresden}
\usecolortheme{beaver}
\usepackage{textcomp}
\usepackage{color}
\usepackage{colortbl}
\title{An experimental study of the learnability of congestion control}
\author{Anirudh~Sivaraman, Keith~Winstein, Pratiksha~Thaker, Hari~Balakrishnan}
\institute{MIT CSAIL\vspace{\baselineskip}}
%\\\textcolor{DarkBlue}{http://web.mit.edu/anirudh/www/sdn-data-plane.html}}
%\date{}


\begin{document}

\begin{frame}

\titlepage

\end{frame}

\begin{Large}
\begin{frame}
\frametitle{Designing congestion-control protocols today}
\begin{itemize}
\item Formulate a mental model of the target network and application workload
\item Decide on the protocol's goal
\item Design a protocol to achieve this goal on the target network
\item Can either be implicit or explicit
\end{itemize}
\end{frame}

\begin{frame}
\frametitle{But, the model is always wrong!}
\begin{itemize}
\item Lost throughput due to stochastic loss 
\item Bufferbloat when queues are incorrectly sized
\item Diminished fairness in small-packet regimes
\item Incast in datacenters
\end{itemize}
\end{frame}

\begin{frame}
\frametitle{Our work}
\begin{itemize}
\item Can we formalize this design process?
\item Quantify the consequences of model mismatch?
\end{itemize}
\end{frame}

\begin{frame}
\frametitle{Approach}
\begin{itemize}
\item Specify a \textit{training scenario} for training.
\begin{itemize}
\item Topology
\item Locations of senders and receiver
\item Application workload
\item Buffer size and queuing discipline 
\end{itemize}
\item Specify an \textit{objective function}.
\item Synthesize protocol automatically.
\item Evaluate on a \textit{testing scenario} inside ns-2
\end{itemize}
\end{frame}

\begin{frame}
\frametitle{Automated protocol synthesis}
\begin{itemize}
\item Find best protocol, given an imperfect network model.
\item Unfortunately, problem is NEXP-complete. 
\end{itemize}
\end{frame}

\begin{frame}
\frametitle{Tractable Attempts at Optimal}
\begin{itemize}
\item Rely on Remy~\cite{remy} to produce Tractable Attempts at Optimal (TAO) congestion-control protocols.
\item Approaches upper bounds on throughput and lower bounds on delay.
\end{itemize}
\end{frame}

\begin{frame}
\frametitle{How far off is Remy from the optimal?}
\begin{itemize}
\item Training scenario:
\begin{tabular}{ll}
Link speed & 32 Mbits/sec \\
minimum RTT & 150 ms \\
Topology & Dumbbell \\
Number of senders & 2 \\
Workload & Exponential ON/OFF times (mean 1 sec) \\
Buffer size & 5 BDP \\
Objective function & $\sum$ log(throughput) - log(delay)
\end{tabular}
\item Testing scenario identical to training scenario
\end{itemize}

\end{frame}

\input optimality

\input linkspeed

\input topology

\input compatibility

\input diversity

\begin{frame}
\frametitle{Related Work}
\begin{itemize}
\item Probably approximately correct learning
\item Transfer learning
\item Machine-generated congestion control
\end{itemize}
\end{frame}

\begin{frame}
\frametitle{Limitations and future work}
\begin{itemize}
\item Generalizability to more complex topologies?
\item Better characterization of gap from optimal
\item Do results change if we learn in-network behavior as well?
\item Model mismatches between simulation and the real world
\end{itemize}
\end{frame}
\end{Large}

\end{document}
