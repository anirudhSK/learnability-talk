\documentclass[svgnames]{beamer}
\usetheme{Dresden}
\usecolortheme{beaver}
\usepackage{textcomp}
\usepackage{color}
\usepackage{colortbl}
\title{An experimental study of the learnability of congestion control}
\author{Anirudh~Sivaraman, Keith~Winstein, Pratiksha~Thaker, Hari~Balakrishnan}
\institute{MIT CSAIL\vspace{\baselineskip}}
%\\\textcolor{DarkBlue}{http://web.mit.edu/anirudh/www/sdn-data-plane.html}}
%\date{}


\begin{document}

\begin{frame}

\titlepage

\end{frame}

\begin{Large}
\begin{frame}
\frametitle{Designing congestion-control protocols today}
\begin{itemize}
\item Formulate a mental model of the target network and application workload
\item Decide on the protocol's goal
\item Design a protocol to achieve this goal on the target network
\item Can either be implicit or explicit
\end{itemize}
\end{frame}

\begin{frame}
\frametitle{But, the model is always wrong ...}
\begin{itemize}
\item Lost throughput due to stochastic loss 
\item Bufferbloat when queues are incorrectly sized
\item Diminished fairness and unpredictability in small-packet regimes
\item Incast in datacenters
\end{itemize}
\end{frame}

\begin{frame}
\frametitle{Our work}
\begin{itemize}
\item Can we formalize this design process?
\item Quantify consequences of model mismatch
\end{itemize}
\end{frame}

\begin{frame}
\frametitle{Approach}
\begin{itemize}
\item Specify a \textit{training scenario} for training.
\begin{itemize}
\item Topology
\item Locations of senders and receiver
\item Application workload
\item Buffer size and queuing discipline 
\end{itemize}
\item Specify an \textit{objective function}
\item Synthesize a protocol using an automated protocol-synthesis tool, Remy~\cite{remy}
\item Evaluate on a \textit{testing scenario} inside ns-2
\item Difference between training and testing scenario represents model imperfection
\end{itemize}
\end{frame}

\begin{frame}
\frametitle{Automated protocol synthesis}
\begin{itemize}
\item Find best protocol for an imperfect network model.
\item Problem is NEXP-complete. 
\item Rely on Remy to produce Tractable Attempts at Optimal (TAO) congestion-control protocols.
\end{itemize}
\end{frame}

\begin{frame}
\frametitle{Tractable Attempts at Optimal}
\begin{itemize}
\item Just how suboptimal are these TAO protocols?
\item  
\item
\end{itemize}
\end{frame}

\input optimality

\input linkspeed

\input topology

\input compatibility

\input diversity

\begin{frame}
\frametitle{Related Work}
\begin{itemize}
\item Probably-approximately correct learning
\item Transfer learning
\item Machine-generated congestion control
\end{itemize}
\end{frame}

\begin{frame}
\frametitle{Limitations and future work}
\begin{itemize}
\item Better characterization of optimal protocols
\item Extending protocol-generation to in-network algorithms as well.
\item Characterize model imperfections between simulation and the real world.
\item Why are the results the way they are?
\end{itemize}
\end{frame}
\end{Large}

\end{document}
