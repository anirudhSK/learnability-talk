\documentclass[svgnames]{beamer}
\usetheme{Dresden}
\usecolortheme{beaver}
\usepackage{textcomp}
\usepackage{color}
\usepackage{colortbl}
\usepackage{pbox}
\setbeamerfont{page number in head/foot}{size=\large}
\setbeamertemplate{footline}[frame number]
\title{An experimental study of the learnability of congestion control}
\author{Anirudh~Sivaraman, Keith~Winstein, Pratiksha~Thaker, Hari~Balakrishnan}
\institute{MIT CSAIL\vspace{\baselineskip}}
%\\\textcolor{DarkBlue}{http://web.mit.edu/anirudh/www/sdn-data-plane.html}}
%\date{}


\begin{document}

\begin{frame}

\titlepage

\end{frame}

\begin{Large}
\begin{frame}
\frametitle{Protocol design in practice}
\begin{itemize}
\item<1-> Implicitly assume certain network conditions
\item<2-> Implicit performance goals
\end{itemize}
\end{frame}

\begin{frame}
\frametitle{But, assumptions are always wrong}
\begin{itemize}
\item<1-> Stochastic loss in wireless networks
\item<2-> Bufferbloat in cellular networks
\item<3-> Incast in datacenter networks
\end{itemize}
\end{frame}

\begin{frame}
\begin{center}
How easy is it to “learn” a network protocol to achieve desired goals, given
an imperfect set of assumptions?
\end{center}
\end{frame}

\begin{frame}
%TODO: Can throw in figure here per D. Wetherall.
\frametitle{Mechanizing protocol design}
\begin{itemize}
\item <1->Supply training networks to automated protocol synthesis tool (Remy, SIGCOMM 2013) 
\item <2->Evaluate synthesized protocol on test networks
\end{itemize}
\end{frame}

\input optimality

\input linkspeed

\input multiplexing

\input topology 

\input compatibility

\begin{frame}
\frametitle{Related Work}
\begin{itemize}
\item Probably approximately correct learning \textbf{(Valiant, 1984)})
%\item Transfer learning \textbf{(Pan \& Yang, 2010)}
\item Stochastic network utility maximization \textbf{(Yi \& Chiang, 2008)}
\end{itemize}
\end{frame}

\begin{frame}
\frametitle{Summary of findings}
\begin{itemize}
\item<1-> Possible to design for 1000x link speed range
\item<2-> Simplifying topology doesn't hurt performance
\item<3-> TCP compatibility is a double-edged sword 
\item<4-> Tension between low and high degrees of multiplexing
\end{itemize}
\end{frame}

\begin{frame}
\frametitle{Limitations}
\begin{itemize}
\item<1-> Very simple experiments
\item<2-> Results could change with better protocol-synthesis tools
\item<3-> Do not model real-world imperfections
\end{itemize}
\end{frame}
\begin{frame}

\frametitle{Ongoing work}
\begin{itemize}
\item<1-> Automated synthesis of datacenter congestion control
\item<2-> Automated synthesis of wide-area congestion control
\end{itemize}
\end{frame}

\begin{frame}
\frametitle{Conclusion and outlook}
\begin{itemize}
\item<1-> Application needs and network conditions evolve ...
\item<2-> ... requiruing us to design new transport protocols.
\item<3-> How hard is it to automate this process like other disciplines have done?
%\item<4-> Paper available at: http://web.mit.edu/anirudh/www/learnability-sigcomm2014.pdf
\end{itemize}
\end{frame}

\end{Large}

\begin{frame}
\frametitle{Backup slides}
\end{frame}
\input caveats

\input remy


\input topology
\input diversity
\end{document}
