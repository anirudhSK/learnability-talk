\documentclass[svgnames]{beamer}
\usetheme{Dresden}
\usecolortheme{beaver}
\usepackage{textcomp}
\usepackage{color}
\usepackage{colortbl}
\usepackage{pbox}
\title{An experimental study of the learnability of congestion control}
\author{Anirudh~Sivaraman, Keith~Winstein, Pratiksha~Thaker, Hari~Balakrishnan}
\institute{MIT CSAIL\vspace{\baselineskip}}
%\\\textcolor{DarkBlue}{http://web.mit.edu/anirudh/www/sdn-data-plane.html}}
%\date{}


\begin{document}

\begin{frame}

\titlepage

\end{frame}

\begin{Large}
\begin{frame}
\frametitle{Designing congestion-control protocols today}
\begin{itemize}
\item<2-> Set of rules for the end points to follow
\item<3-> Implicitly assume certain network conditions
\end{itemize}
\end{frame}

\begin{frame}
\frametitle{Assumptions may break with time}
\begin{itemize}
\item Lost throughput due to stochastic loss 
\item Bufferbloat when queues are incorrectly sized
\item Incast in datacenters
\item<2-> What assumptions matter and what don't?
\end{itemize}
\end{frame}

\begin{frame}
\frametitle{Our work}
\begin{itemize}
\item<1-> Can we formalize this design process ...
\item<2-> to quantify: how easy is it to “learn” a network protocol to achieve desired goals, given
an imperfect model of the network?
\end{itemize}
\end{frame}

\begin{frame}
\frametitle{Contributions}
\begin{itemize}
\item Formalize learnability in the context of congestion control
\item Use it to answer:
\begin{itemize}
\item Do we need to know the link speed exactly?
\item What is the cost of backwards compatibility?
\item Do we need to know the topology exactly?
\end{itemize}
\end{itemize}
\end{frame}

\begin{frame}
\frametitle{Approach}
\begin{itemize}
\item Specify a \textit{training scenario}.
\begin{itemize}
\item Topology
\item Locations of senders and receiver
\item Application workload
\item Buffer size and queuing discipline 
\end{itemize}
\item Specify an \textit{objective function}.
\item Synthesize protocol automatically.
\item Evaluate on a \textit{testing scenario} inside ns-2
\end{itemize}
\end{frame}

\begin{frame}
\frametitle{Formalizing the design process}
\begin{itemize}
\item Find best protocol, given an imperfect network model
\item The problem is hard to solve in general
\item<2-> Compromise: Rely on Remy\footnote<2->{Keith Winstein and Hari Balakrishnan, \textbf{TCP ex Machina: Computer-Generated Congestion Control}, \textit{SIGCOMM 2013}} instead.
\end{itemize}
\end{frame}

\input caveats

\input optimality

\input linkspeed

\input compatibility

\input topology


\begin{frame}
\frametitle{Related Work}
\begin{itemize}
\item Probably approximately correct learning \textbf{(Valiant, 1984)})
\item Transfer learning \textbf{(Pan \& Yang, 2010)}
\item Stochastic network utility maximization \textbf{(Yi \& Chiang, 2008)}
\end{itemize}
\end{frame}

\begin{frame}
\frametitle{Limitations and future work}
\begin{itemize}
\item Generality on delay, multiplexing axis
\item More complex topologies
\item Do results change if we learn in-network behavior as well?
\item Implement at Google (Summer 2014)
\end{itemize}
\end{frame}
\end{Large}

\begin{frame}
\frametitle{Backup slides}
\end{frame}
\input diversity
\end{document}
