\begin{frame}
\frametitle{Learning network protocols despite imperfect assumptions}
\begin{centering}
Is it possible to perform well under both low and high degrees of multiplexing?
\end{centering}
\end{frame}

\begin{frame}
\frametitle{Performance under different degrees of multiplexing}
\begin{centering}

\noindent \only<1>{\includegraphics[width=3.1 in]{muxing-base.pdf}}\only<2>{\includegraphics[width=3.1 in]{muxing-omniscient.pdf}}\only<3>{\includegraphics[width=3.1 in]{muxing-Cubic.pdf}}\only<4>{\includegraphics[width=3.1 in]{muxing-Cubic-over-sfqCoDel.pdf}}\only<5>{\includegraphics[width=3.1 in]{muxing-Tao-1-2.pdf}}\only<6>{\includegraphics[width=3.1 in]{muxing-Tao-1-10.pdf}}\only<7>{\includegraphics[width=3.1 in]{muxing-Tao-1-20.pdf}}\only<8>{\includegraphics[width=3.1 in]{muxing-Tao-1-50.pdf}}\only<9>{\includegraphics[width=3.1 in]{muxing-Tao-1-100.pdf}}
\only<10>{Tension between low and high degrees of multiplexing}

\end{centering}
\end{frame}
